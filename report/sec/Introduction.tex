The goal of this project is to explore the possibility of making a cube balance on its edges by utilizing reaction wheels. Additionally, there are some key areas that were focused on in this project in order for the participants to gather additional experience. These are the following:

\begin{itemize}
    \item Working with a overarching system perspective.
    \item Producing prototypes by means of 3D printing.
    \item Writing a functioning control algorithm tailored to the problem. 
    \item Creating a working program for the entire system.
\end{itemize}

\noindent
The project was divided into three subtasks where the first was to balance an inverted pendulum on a single axis setup, the second was to balance a cube on an edge and the third was to have the cube rise to an edge from rest.

\subsection{Previous work}
This project has been a replica of the work done by M. Gajamohan et. al\cite{cubli-planar}. Their work has shown that it is possible to make a cube balance using reaction wheels and it also a source of information in which the group could to some extent find relevant equations for calculations. However, it must be mentioned that a Google search also indicates Gajamohan et al. are not the only group to have done this kind of project. Other sources of inspiration are the many open source projects from ReM-RC\cite{remrc} and this YouTube video\cite{LEGO-video}.

\subsection{Concept of reaction wheels}
Reaction wheels are a motor-controlled variant of flywheels, i.e. free spinning wheels used to store rotational energy by conserving angular momentum. When the rotational speed of the wheel changes, the body its attached to will counter-rotate due to the conservation of angular momentum. Note that rotation of the body can only happen during torque, meaning that accelerating a still wheel has the same effect as braking an already spinning wheel. By attaching a motor to the fly wheel these torques can be controlled, creating a reaction wheel. One of the most common applications of reaction wheels is in spacecrafts and satellites, where multiple wheels are used to control and stabilize the orientation.