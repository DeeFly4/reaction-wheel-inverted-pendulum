Our supervisor said during a meeting that "There are two ways of solving a control problem. Either, one designs the system and actively chooses which components are to be used, or one is given a complete system and has to do a lot more work on control modelling and algorithms."

\subsection{Limitations in the simulation}
There were a number of limitations in the simulation that has to be addressed. First of all, the simulation uses the linearized model in \eqref{eq:state-space}. This does not completely represent the reality of the dynamics of the single axis setup. However, since the working range of the angle of the system is only a couple of degrees, it was reasonable to use the linearized model. Second of all, many of the parameters used in the model were approximated. The moments of inertia were all derived using standard formulas and could have been more accurate. Another example is the friction coefficient, which was taken from \cite{cubli-planar}. Although there are a number of limitations to the simulation, the purpose of it was not to get a working controller that could be directly applied to the real model, but instead to see how a PID controller compares to a LQG controller, and which parameters are important.

\subsection{Interpreting the simulations}
As seen in Figures~\ref{fig:PIDGraph} and \ref{fig:LQGGraph}, both controllers could stabilize the system with the given set of parameters. They could also handle an impulse disturbance of $0.5\degree$. When comparing the two controllers, one instantly notes the smoother response of the LQG controller. However, the LQG is highly dependent on a good model of the system and to what extent our model and parameters are sufficient, will not be answered in this paper, since a LQG controller was not implemented in the real system. Both controllers in the simulation used a sampling time of 15 ms. This was done in order to insure that the sampling time was not impossible to meet.
\\\\
As mentioned in Table~\ref{table:ParamSig}, the most significant parameters were the sampling time, the maximum torque that the motor can generate and the length to center of mass from the pivot point. This is reasonable since a longer length to the center of mass will increase the torque required cancel the gravitational force. The motor torque is also important to counter the torque generated by the body itself. It is also reasonable that the sampling time is significant, since  even a small change in angle will have large effect on the necessary torque to counter the fall and the change in angle should be measured as fast as possible. Also, the mass of the frame and the radius of the wheel, seemed less important. This is likely due to the fact that the mass of the frame is already low, and the same applies to the length to center of mass. The same applies to the friction coefficient and the mass of the wheel, which both already had low values. It is important to note that this observation was only done qualitatively, since the simulation is based on approximations. The purpose was to observe the trends mentioned above rather than finding the exact parameters that can be handled. With this said, it would be recommended to initially lower the sampling time, increase the maximum torque and diminish the length to center of mass, before changing anything else. 
\\\\
Comparing the PID parameters in Table~\ref{table:PIDParameters} with the real ones, one notices a large difference. This is because of a number of reasons. One being that the control signal is very different when comparing Figure~\ref{fig:PIDGraph} with Figure~\ref{fig:physical-balance}, since the simulation input signal is in ampere and the code for the real process is the velocity for the wheel. Also, the simulation inputs angles in radiance, while the real process inputs angles in degrees. Another reason why the PID parameters differ, is because the physical parameters differ between the simulation and the real process. This is true, since the real setup used is not the same as in Figure~\ref{fig:cubli_planar_diagram}, which is the structure simulated. Also, most parameters were identified and can differ from the real physical parameters.

\subsection{Tuning the PID controller on the single axis setup}
The PID controller was tuned through trial and error with the help of some intuitive reasoning. Most of the early tuning was done with only PD-control. Due to the system's instability there can be no steady state error, leading us to believe no integral action should be used. But these ideas proved to be flawed.
\\\\
Firstly, while it is true that we may want to ease up on the torque when the pendulum is already on its way up, that does not mean to lower the control signal. Remember that the control signal is (roughly) proportional to speed so a lower control signal actually means decelerating the wheel. If the wheel decelerates while the pendulum is rising the effect is a torque going the opposite way, sending the pendulum back down. So a too large $K_D$ may hurt the balancing.
\\\\
Secondly, by introducing integral action we can get the desired behavior of increasing the speed, which maintains torque, even when the pendulum is rising. The introduction of integral action is likely also why the control signal oscillates around a positive base speed instead of 0. It is then important to have integral action that reacts quickly, but does not grow too large. This is why setting $I_{\text{max}} < u_{\text{max}}$ and adding reference adjustment eventually led to the best results. Another advantage of reference adjustment is that it allows the process to find its own equilibrium. A very interesting example of this can be found in \cite{LEGO-video} at 11:25. If there is a discrepancy between what the IMU registers as $0\degree$ and the physical equilibrium, it could be hard to achieve balance if always controlling to $r=0$. This explains the results seen in Figure~\ref{fig:physical-balance}. The reference adjustment turned out to be crucial for this reason.

\subsection{IMU signal}
Surprisingly, when measuring the output angle from the IMU we could not detect any noise at all (Figure~\ref{fig:imu_angle}). Despite this the raw data readings from the gyrometer were very noisy (Figure \ref{fig:imu_gyro}). This is problematic if implementing state feedback in the microcontroller. A good observer would then be needed, consequently requiring a more accurate model to develop this observer with the help of simulations. An alternate way of estimating $\dot\theta_b$ is of course with a difference quotient, as in the discretized PID controller. We considered using the gyrometer readings as a more accurate estimate of the derivative even with PID control, but after discovering the noisiness we decided against it.

\subsection{Weight reduction}
In cases where more time is available, the final design can be developed through several iterations. For this project, time was sufficient for designing a single axis setup as a pilot study for the actual cube. While the thickness of the cube frame was heavily reduced, compared to the previous prototype, at points where it could be done, after printing it was concluded that the thickness could have been reduced even more in order to save weight. Although, the group is concerned if such a weight reduction would be enough for the motors to be strong enough to balance the cube. The components on their own also contributed to the total weight. With this in mind, even stronger motors would almost certainly be needed if the cube is to be driven by a large battery like in this case.